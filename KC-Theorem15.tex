%
% To use this as a template for turning in your solutions, change the flag
% \inclsolns from 0 to 1. Make sure you include macros.tex in the directory
% containing this file. Edit the "author" and "collaborators" fields as
% appropriate. Write your solutions where indicated.
%

\documentclass[12pt]{article}

\usepackage{cite}
\usepackage{fullpage}
\usepackage{graphicx}
\usepackage{enumerate}
\usepackage{comment}
\usepackage[hidelinks]{hyperref}
\usepackage{amsmath, amsfonts, amsthm, amssymb, algorithm, graphicx, mathtools,xfrac}
\usepackage[noend]{algpseudocode}

\input{macros}

\theoremstyle{definition}

\makeatletter
\def\collaborators#1{\gdef\@collaborators{#1}}
\def\@collaborators{\@latex@warning@no@line{No \noexpand\collaborators given}}
\makeatother


\begin{document}

\smallskip
\smallskip
\smallskip

\begin{center}

{\Large Review on Circuit Minimization Problem by Kabanets-Cai}

\bigskip
{\normalsize Date: \today}

\bigskip
{\normalsize (First Draft)}

\end{center}

\section{Introduction}

In the Minimum Circuit Size Problem (MCSP), we are given a truth table of some Boolean function together with a positive integer $s_n$ as input, and our task is to answer the question whether there exists a circuit of size at most $s_n$ that computes the function represented by the given truth table. Formally speaking, given a Boolean function $f_n: \{0, 1\}^n \rightarrow \{0, 1\}$, an instance of $\mathprob{MCSP}$ is a tuple $\langle T_n, s_n \rangle$ where $T_n$ is the string of length $2^n$
representing the truth table of $f_n$ and $f_n$ is computable by a circuit of size at most $s_n$. It is easy to see that $\mathprob{MCSP}$ is in $\NP$. Namely, we can define a certificate as some proposed circuit $C$ of size at most $s_n$, and verify whether $C$ computes each entry of the truth table correctly in polynomial time. With that being said, a natural question arises: Is $\mathprob{MCSP}$  $\NP$-complete?

{
% we can use this structure if you like it:
\color{gray}
\renewcommand{\arraystretch}{1.5}
\begin{center}
  \begin{tabular}{|p{2cm}p{11cm}|}
    \hline
    Problem:
    &
    $\mathprob{MSCP}$
    \\
    \hline
    Input:
    &
    A tuple $\langle T_n, s_n \rangle$ consisting of a truth table $T_n$ for
    a Boolean function of arity $n$ and an integer $s_n$
    \\
    Question: & Is there a circuit $C_n$ of size at most $s_n$ computing $T_n$?
    \\
    \hline
  \end{tabular}
\end{center}
}

In their paper ``Circuit Minimization Problem,'' Valentine Kabanets and Jin-Yi Cai addressed the difficulty of showing $\mathprob{MCSP}$ to be $\NP$-hard. Namely, they showed some consequences that are unlikely to happen if there exists a polynomial-time reduction $R$ from $\mathprob{SAT}$ to $\mathprob{MCSP}$ that is ``natural,'' in the sense that the size of the output depends on the size of the inputs only, and these sizes are polynomially related.  Furthermore, the authors pointed out why it is challenging to prove $\mathprob{MCSP} \notin \P$ (again, by providing some consequences whose likelihood are questionable). Clearly, such proof would imply $\P \neq \NP$ which goes beyond the currently known techniques. 

\textbf{The rest of the review:} In Section 2, we will provide some definitions required to understand the main results and theorems of the paper. Section 3 introduces some main consequences when $\mathprob{MCSP} \notin \P$ and $\mathprob{MCSP}$ is $\NP$-hard under ``natural'' reductions. Finally, we conclude the review by providing some insightful remarks and directions for further research on this topic in Section 4.

\bigskip
\section{Some definitions:}
\begin{enumerate} [1.]
	\item The class Sub-Exponential: $\class{SUBEXP} = \bigcap_{\epsilon > 0} \TIME(2^{n^{\epsilon}})$

  \item The class $\class{QP}$ of languages decided by a TM in
    \emph{quasi-polynomial} time defined by
    \[
      \class{QP}
      \coloneqq
      \TIME(n^{\mathrm{polylog}(n)})
      =
      \bigcup_{c > 1}\TIME(2^{\log^c n})
      =
      \bigcup_{c > 1}\TIME(n^{\log^c n})
    \]
    where we obtain the last equality since
    \[
      2^{(\log n)^{c+1}} = \exp( \log^{c+1} n ) = \exp( \log n \log^c n )
      = n^{\log^c n} \,.
    \]
    Note that $\class{QP}$ contains $\P$ since $\mathrm{polylog}(n) \in \Omega(1)$.

	\item The class exponential time with linear exponential is defined as
    $\class{E} \coloneqq \TIME(2^{O(n)})$.
    Since $\log^c n \in O(n)$ for any $c > 0$, we obtain that
    $\class{QP} \subseteq \class E$.
	
	\item \textbf{Natural Reduction}: For two problems $A$ and $B$ and a Karp reduction from $A$ to $B$, we say the reduction $R$ is natural if, for any instance $I$ of $A$, the length of the output $|R(I)|$, as well as all possible output parameters $s_n$, depend only on the input length $|I|$. Furthermore, $|I|$ and $|R(I)|$ are polynomial related.
	
	Example: $\mathprob{SAT} \leq_p \mathprob{3SAT}$ is natural. 
	
	Question(s): What is output parameter $s_n$? Example?
\end{enumerate}

\newpage

\begin{lemma}
  \label{lem:ph-sub-qp}
  If $\NP \subseteq \class{QP}$ then
  $\class{PH} \subseteq \class{QP}$.
\end{lemma}

\begin{proof}
  We first show that $\NP = \exists \P \subseteq \class{QP}$ also implies that
  $\exists \class{QP} \subseteq \class{QP}$ by a padding
  argument.\footnote{The class $\exists \class{QP}$ is defined to consist of
  all languages $L$ such that there exists a quasi-polynomial TM $M$ and a
  polynomial $p$ with $x \in L \iff \exists y \in \{0,1\}^{p(|x|)}\
  M(x,y) = 1$}
  To this end, assume that $L \in \exists \class{QP}$ such that it is decided
  by a $\exp(\log^c n)$-time verifier $M$.
  We define our padded language as
  $L_{\textrm{pad}} \coloneqq \set{ x 1^{\exp(\log^c n)} \mid x \in L }$.
  We obtain that $L_{\textrm{pad}} \in \exists \P$ since applying $M$ to a
  padded element takes takes only linear time.
  It follows by our assumption that $L_{\textrm{pad}} \in \class{QP}$;
  say $L_{\textrm{pad}}$ is decided by an $\exp(\log^{c'} n)$-time TM $M'$.
  Now, the key idea lies in showing that this already implies
  $L \in \class{QP}$:
  Let $x \in \{0,1\}^*$ be of length $n \coloneqq |x|$ and
  its padded version $y \coloneqq x 1^{\exp(\log^c n)}$ of length
  $m = \exp(\log^c n)$. Then, the time needed by $M'$ to decide $y$ is
  \[
    \exp(\log^{c'} m)
    =
    \exp(\log^{c'}(\exp(\log^c n)))
    =
    \exp(\log^{c \cdot c'} n)
  \]
  which is again quasi-polynomial in $n$.
  Therefore, we can decide $x \in L$ by applying padding and running $M'$ all
  in quasi-polynomial time.

  The second step is to show that also
  $\forall \class{QP} \subseteq \class{QP}$.
  By definition, given a language $L \in \forall \class{QP}$, there exists a
  polynomial $p$ and a quasi-polynomial time verifier $M$ such that
  \[
    x \in L \iff \forall y \in \{0,1\}^{p(|x|)}\ M(x,y) = 1 \,.
  \]
  However, the right hand side is equivalent to the negation of
  $(\exists y \in \{0,1\}^{p(|x|)}\ M(x,y) = 0)$, a statement which can itself
  be decided in quasi-polynomial time by the above argument.
  Since all quasi-polynomial time TMs are also deterministic, we can also
  negate the output efficiently. It follows that $L \in \class{QP}$.

  By an inductive argument, we see that $\class{PH}$ collapses into a subset of
  $\class{QP}$.
\end{proof}

\begin{lemma}
  \label{lem:qp-super-circ}
  $\class{QP}^{\class{\Sigma}^p_k}$ contains a language which does not belong
  to $\Ppoly$ for some $k \in \N$.
\end{lemma}

\begin{proof}
  The proof follows a nonuniform diagonalization argument. We first define
  a language which will be hard to compute for any polynomial-size circuit
  family:
  Let $L'$ be the language consisting of tuples
  $\langle x, 1^{\exp(\log^3 n)} \rangle$
  with $n \coloneqq |x|$
  such that $C(x) = 1$ where $C$ is the
  lexicographically first circuit of size $\exp(\log^3 n)$ which is not
  computed by any circuit of size $\exp(\log^2 n)$.
  The existence of such a circuit for sufficiently large $n$ follows from a
  slightly more careful analysis of the nonuniform hierarchy theorem
  (Theorem~6.22).

  We can decide membership $\langle x, 1^{\exp(\log^3 n)} \rangle \in L'$
  by a $\class{\Sigma}^p_4$-oracle as in Problem~(1c) of Homework~7.
  Finally, we define our language $L$ of superpolynomial circuit complexity
  as the output of a $\class{QP}^{\class{\Sigma}^p_4}$-machine: Given an input
  $x \in \{0,1\}^n$, query the oracle for $L'$ with
  $\langle x, 1^{\exp(\log^3 n)} \rangle$ and output its answer.

  We constructed this language such that it is hard for a polynomial-size
  circuit to compute. To see this, assume to the contrary that there is
  a $n^a$-size circuit family.
  However, since $n^a \in o(\exp(\log^2 n))$ and we particularly excluded any
  circuits of size less than $\exp(\log^2 n)$, this is a contradiction.
\end{proof}

\bigskip

\begin{theorem}[Theorem 15]
If $\mathprob{MCSP}$ is $\NP$-hard under a natural reduction from $\mathprob{SAT}$, then
\begin{enumerate} [1.]
	\item $\class{E}$ contains a family of Boolean functions $f_n$ not in $\Ppoly$ (i.o.), and
	\item $\class{E}$ contains a family of Boolean functions $f_n$ of circuit complexity $2^{\Omega(n)}$ (i.o.), unless $\NP \subseteq \class{SUBEXP}$
\end{enumerate}
\end{theorem}

\begin{proof}
\begin{enumerate}[-]
	\item Statement 1: consider two cases
	\begin{enumerate} [+]
		\item Case 1: $\NP \subseteq \class{QP}$
		
      {\color{gray}
      It is obvious to see that $\NP \subseteq \class{QP} \implies \class{PH} \subseteq \class{QP}$ (think of this as a generalisation of $\NP \subseteq \P \implies \class{PH} \subseteq \P$). Also, we make an observation that $\class{QP}^{\Sigma_k^p}$, for some $k \in \N$, contains a language of superpolynomial circuit complexity. The proof of this claim is as follow: (help!!!!)
		
		Combining these two results, we get the following $\NP \subseteq \class{QP} \implies \class{QP}^{\class{PH}} \subseteq \class{QP}^{\class{QP}} \subseteq \class{QP} \subset \class{E}$ contains a family of function not in $\Ppoly$.}

      Applying Lemma~\ref{lem:ph-sub-qp} to this assumption yields that
      $\class{QP}^{\class{PH}} \subseteq \class{QP}^{\class{QP}} \subseteq
       \class{QP} \subseteq \class{E}$.
      Lemma~\ref{lem:qp-super-circ} shows that also $\class E$ contains
      a language of superpolynomial circuit complexity.

      % how does
      %   'a language of superpolynomial circuit complexity'
      % relate to
      %   'a family of Boolean functions not in P/poly'
      % ?
		
		Hence, $\class{E} \not\subseteq \Ppoly$\\
		
		\item Case 2: $\NP \not\subseteq \class{QP}$
		
		A given natural reduction $R$ from $\mathprob{SAT}$ to $\mathprob{MCSP}$ maps every family of boolean formulas of size $n$  to truth tables of Boolean functions on $k = \theta(\log n)$ variables (is the symbol $\theta$ significant? more elaboration on $\theta(\log n)$, like why is that?) and a parameter $s_n$.
		
		Since the reduction is natural, $s_n$ is some function of $n$ only and thus it could be some fixed polynomial $\log^c n$. Since there are at most [$\dots$] (fill in the gap, I don't really understand why they said there are $n^{\text{polylog(n)}}$ circuits) that many different circuits on $k$ inputs given $\log^c n$ gates. Thus, all such instances of $\mathprob{MCSP}$ where $s_n = \log^c n$, for some constant $c$, are solvable in quasi-polynomial time. This implies that $\mathprob{SAT} \in \class{QP}$ which is a contradiction to our assumption since $\mathprob{SAT}$ is $\NP$-complete. (should we say something about the relationship between $k$ and $\log n$? My understanding is that what we have shown here implies that the circuit size $s_n \neq \log^c n$ which is related to something like $k^c$, i.e. can't get a circuit of poly(k) size which leads to the thing below)
		
		Therefore, we can obtain the desired k-variable functions not in $\Ppoly$ by applying the reduction $R$ to any trivial family of unsatisfiable formulas. Clearly, this family of hard functions is computable in time $2^{O(k)}$. (maybe more elaboration on this too...).
		
	\end{enumerate}
	\item Statement 2: the proof is similar (you want to do it too?)
\end{enumerate}
\end{proof}

\end{document}






