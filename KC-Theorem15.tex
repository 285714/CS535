%
% To use this as a template for turning in your solutions, change the flag
% \inclsolns from 0 to 1. Make sure you include macros.tex in the directory
% containing this file. Edit the "author" and "collaborators" fields as
% appropriate. Write your solutions where indicated.
%

\documentclass[12pt]{article}

\usepackage{cite}
\usepackage{fullpage}
\usepackage{graphicx}
\usepackage{enumerate}
\usepackage{comment}
\usepackage{hyperref}
\usepackage{amsmath, amsfonts, amsthm, amssymb, algorithm, graphicx, mathtools,xfrac}
\usepackage[noend]{algpseudocode}

\input{macros}

\theoremstyle{definition}

\makeatletter
\def\collaborators#1{\gdef\@collaborators{#1}}
\def\@collaborators{\@latex@warning@no@line{No \noexpand\collaborators given}}
\makeatother


\begin{document}

\smallskip
\smallskip
\smallskip

\begin{center}

{\Large Notes on Circuit Minimization Problem by Kabanets-Cai}

\bigskip
{\normalsize Date: 11/10/2020}

\bigskip
{\normalsize (First Draft)}

\end{center}

\textbf{MCSP:} the input is a truth table (of length $2^n$) representing a Boolean function $f_n: \{0, 1\}^n \rightarrow \{0, 1\}$ and a number $s_n \in \N$ (encoded in binary). Question: is $f_n$ computable by a Boolean circuit of size at most $s_n$?

\bigskip
\textbf{Natural Reduction:} For two problems $A$ and $B$ and a Karp reduction from $A$ to $B$, we say the reduction $R$ is natural if, for any instance $I$ of $A$, the length of the output $|R(I)|$, as well as all possible output parameters $s_n$, depend only on the input length $|I|$. Furthermore, $|I|$ and $|R(I)|$ are polynomial related.

Example: $\mathprob{SAT} \leq_p \mathprob{3SAT}$ is natural. 

Question(s): What is output parameter $s_n$? Example?

\bigskip
\textbf{Some definitions:}
\begin{enumerate} [1.]
	\item The class Sub-Exponential: $\class{SUBEXP} = \bigcap_{\epsilon > 0} \TIME(2^{n^{\epsilon}})$
	\item The class quasi-polynomial: $\class{QP} = \bigcup_{c > 1}\TIME(2^{\log^c n}) = \bigcup_{k = c-1}\TIME(n^{\log^k n})$
	\item The class exponential time with linear exponential: $\class{E} = \TIME(2^{O(n)})$
\end{enumerate}

\bigskip
\textbf{Theorem 15:} If $\mathprob{MCSP}$ is $\NP$-hard under a natural reduction from $\mathprob{SAT}$, then
\begin{enumerate} [1.]
	\item $\class{E}$ contains a family of Boolean functions $f_n$ not in $\Ppoly$ (i.o.), and
	\item $\class{E}$ contains a family of Boolean functions $f_n$ of circuit complexity $2^{\Omega(n)}$ (i.o.), unless $\NP \subseteq \class{SUBEXP}$
\end{enumerate}

\textbf{Proof:}
\begin{enumerate}[-]
	\item Statement 1: consider two cases
	\begin{enumerate} [+]
		\item Case 1: $\NP \subseteq \class{QP}$
		
		It is obvious to see that $\NP \subseteq \class{QP} \implies \class{PH} \subseteq \class{QP}$ (think of this as a generalisation of $\NP \subseteq \P \implies \class{PH} \subseteq \P$). Also, we make an observation that $\class{QP}^{\Sigma_k^p}$, for some $k \in \N$, contains a language of superpolynomial circuit complexity. The proof of this claim is as follow: (help!!!!)
		
		Combining these two results, we get the following $\NP \subseteq \class{QP} \implies \class{QP}^{\class{PH}} \subseteq \class{QP}^{\class{QP}} \subseteq \class{QP} \subset \class{E}$ contains a family of function not in $\Ppoly$. 
		
		Hence, $\class{E} \not\subseteq \Ppoly$\\
		
		\item Case 2: $\NP \not\subseteq \class{QP}$
		
		A given natural reduction $R$ from $\mathprob{SAT}$ to $\mathprob{MCSP}$ maps every family of boolean formulas of size $n$  to truth tables of Boolean functions on $k = \theta(\log n)$ variables (is the symbol $\theta$ significant? more elaboration on $\theta(\log n)$, like why is that?) and a parameter $s_n$.
		
		Since the reduction is natural, $s_n$ is some function of $n$ only and thus it could be some fixed polynomial $\log^c n$. Since there are at most [$\dots$] (fill in the gap, I don't really understand why they said there are $n^{\text{polylog(n)}}$ circuits) that many different circuits on $k$ inputs given $\log^c n$ gates. Thus, all such instances of $\mathprob{MCSP}$ where $s_n = \log^c n$, for some constant $c$, are solvable in quasi-polynomial time. This implies that $\mathprob{SAT} \in \class{QP}$ which is a contradiction to our assumption since $\mathprob{SAT}$ is $\NP$-complete. (should we say something about the relationship between $k$ and $\log n$? My understanding is that what we have shown here implies that the circuit size $s_n \neq \log^c n$ which is related to something like $k^c$, i.e. can't get a circuit of poly(k) size which leads to the thing below)
		
		Therefore, we can obtain the desired k-variable functions not in $\Ppoly$ by applying the reduction $R$ to any trivial family of unsatisfiable formulas. Clearly, this family of hard functions is computable in time $2^{O(k)}$. (maybe more elaboration on this too...).
		
	\end{enumerate}
	\item Statement 2: the proof is similar (you want to do it too?)
\end{enumerate}

\end{document}






