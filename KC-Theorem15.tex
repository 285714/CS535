%
% To use this as a template for turning in your solutions, change the flag
% \inclsolns from 0 to 1. Make sure you include macros.tex in the directory
% containing this file. Edit the "author" and "collaborators" fields as
% appropriate. Write your solutions where indicated.
%

\documentclass[12pt]{article}

\usepackage{cite}
\usepackage{fullpage}
\usepackage{graphicx}
\usepackage{enumerate}
\usepackage{comment}
\usepackage[hidelinks]{hyperref}
\usepackage{amsmath, amsfonts, amsthm, amssymb, algorithm, graphicx, mathtools,xfrac}
\usepackage[noend]{algpseudocode}


\usepackage{amsmath}
\usepackage{amssymb}



\newtheorem{theorem}{Theorem}
\newtheorem{conjecture}[theorem]{Conjecture}
\newtheorem{lemma}[theorem]{Lemma}
\newtheorem{proposition}[theorem]{Proposition}
\newtheorem{corollary}[theorem]{Corollary}
\newtheorem{claim}[theorem]{Claim}
\newtheorem{fact}[theorem]{Fact}
\newtheorem{openprob}[theorem]{Open Problem}
\newtheorem{remk}[theorem]{Remark}
\newtheorem{apdxlemma}{Lemma}
\theoremstyle{definition}
\newtheorem{definition}[theorem]{Definition}
\newtheorem{example}[theorem]{Example}
\newtheorem{problem}{Problem}

\newenvironment{remark}{\begin{remk}

\begin{normalfont}}{\end{normalfont}
\end{remk}}
\newtheorem{sublemma}[theorem]{Sublemma}


%%%%%%%%%%%%%%%%%%%%%%% text macros
\newcommand{\etal}{{\it et~al.\ }}
\newcommand{\ie} {{\it i.e.,\ }}
\newcommand{\eg} {{\it e.g.,\ }}
\newcommand{\cf}{{\it cf.,\ }}

%%%%%%%%%%%%%%%%%%%%%%% general useful macros
\newcommand{\eqdef}{\mathbin{\stackrel{\rm def}{=}}}
\newcommand{\R}{{\mathbb R}}
\newcommand{\N}{{\mathbb{N}}}
\newcommand{\Z}{{\mathbb Z}}
\newcommand{\F}{{\mathbb F}}
\newcommand{\poly}{{\mathrm{poly}}}
\newcommand{\polylog}{{\mathrm{polylog}}}
\newcommand{\loglog}{{\mathop{\mathrm{loglog}}}}
\newcommand{\zo}{\{0,1\}}
\newcommand{\suchthat}{{\;\; : \;\;}}
\newcommand{\pr}[1]{\Pr\left[#1\right]}
\newcommand{\deffont}{\em}
\newcommand{\getsr}{\mathbin{\stackrel{\mbox{\tiny R}}{\gets}}}
\newcommand{\E}{\mathbb{E}}
\newcommand{\Var}{\mathop{\mathrm Var}\displaylimits}
\newcommand{\eps}{\varepsilon}
\newcommand{\one}{\mathbf{1}}

%%%%%%%%%%%%%%%%%%%%%%%%%% linear algebra
\newcommand{\rank}{\operatorname{rank}}
\newcommand{\spn}{\operatorname{span}}

%%%%%%%%%%%%%%%%%%% macros particular to this course

\newcommand{\mathprob}[1]{\mathsf{#1}}
\newcommand{\SAT}{\mathprob{SAT}}
\newcommand{\yes}{{\sc yes}}
\newcommand{\no}{{\sc no}}

\newcommand{\class}[1]{\mathbf{#1}}
\newcommand{\TIME}{\class{DTIME}}
\newcommand{\NTIME}{\class{NTIME}}
\renewcommand{\P}{\class{P}}
\newcommand{\NP}{\class{NP}}
\newcommand{\coNP}{\class{coNP}}
\newcommand{\BPP}{\class{BPP}}
\newcommand{\BQP}{\class{BQP}}
\newcommand{\RP}{\class{RP}}
\newcommand{\coRP}{\class{coRP}}
\newcommand{\ZPP}{\class{ZPP}}
\newcommand{\MA}{\class{MA}}
\newcommand{\AM}{\class{AM}}
\newcommand{\Sigmacc}[1]{\class{\Sigma^p_{#1}}}
\newcommand{\Picc}[1]{\class{\Pi^p_{#1}}}
\newcommand{\SZK}{\class{SZK}}
\newcommand{\PP}{\class{PP}}
\newcommand{\sharpP}{\class{\#P}}
\newcommand{\IP}{\class{IP}}
\newcommand{\Ecc}{\class{E}}
\newcommand{\EXP}{\class{EXP}}
\newcommand{\NEXP}{\class{NEXP}}

\newcommand{\SPACE}{\class{SPACE}}
\newcommand{\NSPACE}{\class{NSPACE}}
\renewcommand{\L}{\class{L}}
\newcommand{\NL}{\class{NL}}
\newcommand{\coNL}{\class{coNL}}
\newcommand{\BPL}{\class{BPL}}
\newcommand{\RL}{\class{RL}}
\newcommand{\PSPACE}{\class{PSPACE}}

\newcommand{\Ppoly}{\class{P_{/poly}}}
\newcommand{\AC}{\class{AC}}
\newcommand{\NC}{\class{NC}}
\newcommand{\ACC}{\class{ACC}}

\newcommand{\PCP}{\class{PCP}}

\newcommand{\negl}{{\mathrm{neg}}}
\newcommand{\sgn}{\operatorname{sgn}}

\newcommand{\Enc}{\operatorname{Enc}}
\newcommand{\Dec}{\operatorname{Dec}}

%%%%%%Gates%%%%%%%%
\newcommand{\THR}{\mathrm{THR}}
\newcommand{\MAJ}{\mathrm{MAJ}}
\newcommand{\AND}{\mathrm{AND}}
\newcommand{\OR}{\mathrm{OR}}
\newcommand{\XOR}{\mathrm{XOR}}

%%%%%%%Functions%%%%%
\newcommand{\EQ}{\mathrm{EQ}}
\newcommand{\GT}{\mathrm{GT}}


%%%%%%%Custom%%%%%
\usepackage{tikz}
\newcommand{\set}[1]{\{\, {#1} \,\}}
\newcommand{\lencode}{\tikz{\draw (0.15,0) -- (0,0) -- (0,0.15)}}
\newcommand{\rencode}{\tikz{\draw (-0.15,0) -- (0,0) -- (0,0.15)}}

% \usepackage{algpseudocode}
% \usepackage[linesnumbered,ruled,vlined]{algorithm2e}



\theoremstyle{definition}

\makeatletter
\def\collaborators#1{\gdef\@collaborators{#1}}
\def\@collaborators{\@latex@warning@no@line{No \noexpand\collaborators given}}
\makeatother


\begin{document}

\smallskip
\smallskip
\smallskip

\begin{center}

{\Large Review on Circuit Minimization Problem by Kabanets-Cai}

\bigskip
{\normalsize Date: \today}

\bigskip
{\normalsize (First Draft)}

\end{center}

\section{Introduction}

In the Minimum Circuit Size Problem (MCSP), we are given a truth table of some Boolean function together with a positive integer $s_n$ as input, and our task is to answer the question whether there exists a circuit of size at most $s_n$ that computes the function represented by the given truth table. Formally speaking, given a Boolean function $f_n: \{0, 1\}^n \rightarrow \{0, 1\}$, an instance of $\mathprob{MCSP}$ is a tuple $\langle T_n, s_n \rangle$ where $T_n$ is the string of length $2^n$
representing the truth table of $f_n$ and $f_n$ is computable by a circuit of size at most $s_n$. It is easy to see that $\mathprob{MCSP}$ is in $\NP$. Namely, we can define a certificate as some proposed circuit $C$ of size at most $s_n$, and verify whether $C$ computes each entry of the truth table correctly in polynomial time. With that being said, a natural question arises: Is $\mathprob{MCSP}$  $\NP$-complete?

{
% we can use this structure if you like it:
\color{gray}
\renewcommand{\arraystretch}{1.5}
\begin{center}
  \begin{tabular}{|p{2cm}p{11cm}|}
    \hline
    Problem:
    &
    $\mathprob{MSCP}$
    \\
    \hline
    Input:
    &
    A tuple $\langle T_n, s_n \rangle$ consisting of a truth table $T_n$ for
    a Boolean function of arity $n$ and an integer $s_n$
    \\
    Question: & Is there a circuit $C_n$ of size at most $s_n$ computing $T_n$?
    \\
    \hline
  \end{tabular}
\end{center}
}

In their paper ``Circuit Minimization Problem,'' Valentine Kabanets and Jin-Yi Cai addressed the difficulty of showing $\mathprob{MCSP}$ to be $\NP$-hard. Namely, they showed some consequences that are unlikely to happen if there exists a polynomial-time reduction $R$ from $\mathprob{SAT}$ to $\mathprob{MCSP}$ that is ``natural,'' in the sense that the size of the output depends on the size of the inputs only, and these sizes are polynomially related.  Furthermore, the authors pointed out why it is challenging to prove $\mathprob{MCSP} \notin \P$ (again, by providing some consequences whose likelihood are questionable). Clearly, such proof would imply $\P \neq \NP$ which goes beyond the currently known techniques. 

\textbf{The rest of the review:} In Section 2, we will provide some definitions required to understand the main results and theorems of the paper. Section 3 introduces some main consequences when $\mathprob{MCSP} \notin \P$ and $\mathprob{MCSP}$ is $\NP$-hard under ``natural'' reductions. Finally, we conclude the review by providing some insightful remarks and directions for further research on this topic in Section 4.

\newpage
Note that for functions $f, g \colon \N_+ \to \N_+$ holds
$\exp(f(n)) \in o(\exp(g(n)))$ if and only if $g(n) - f(n) \to \infty$ for
$n \to \infty$.
This is easily seen by looking at the limit definition of Landau symbols, i.e.
\begin{multline*}
  f(n) \in o(g(n))
  \iff
  0 = \lim_{n \to \infty} \frac{\exp(f(n))}{\exp(g(n))}
  = \lim_{n \to \infty} \exp(f(n) - g(n)) \\
  \iff
  \lim_{n \to \infty} f(n) - g(n) = -\infty \,.
\end{multline*}

\bigskip
\section{Some definitions:}
\begin{enumerate} [1.]
	\item The class Sub-Exponential: $\class{SUBEXP} = \bigcap_{\epsilon > 0} \TIME(2^{n^{\epsilon}})$

  \item The class $\class{QP}$ of languages decided by a TM in
    \emph{quasi-polynomial} time defined by
    \[
      \class{QP}
      \coloneqq
      \TIME(n^{\mathrm{polylog}(n)})
      =
      \bigcup_{c > 1}\TIME(2^{\log^c n})
      =
      \bigcup_{c > 1}\TIME(n^{\log^c n})
    \]
    where we obtain the last equality since
    \[
      2^{(\log n)^{c+1}} = \exp( \log^{c+1} n ) = \exp( \log n \log^c n )
      = n^{\log^c n} \,.
    \]
    Note that $\class{QP}$ contains $\P$ since
    $\mathrm{polylog}(n) \in \Omega(1)$.
    Also, $\class{SUBEXP}$ contains $\class{QP}$ since for every
    $\varepsilon > 0$ and $c > 1$ holds that
    $\exp(\log^c n) \in o(\exp(n^\varepsilon))$.

	\item The class exponential time with linear exponential is defined as
    $\class{E} \coloneqq \TIME(2^{O(n)})$.
    Since $\log^c n \in O(n)$ for any $c > 0$, we obtain that
    $\class{QP} \subseteq \class E$.
	
	\item \textbf{Natural Reduction}: For two problems $A$ and $B$ and a Karp reduction from $A$ to $B$, we say the reduction $R$ is natural if, for any instance $I$ of $A$, the length of the output $|R(I)|$, as well as all possible output parameters $s_n$, depend only on the input length $|I|$. Furthermore, $|I|$ and $|R(I)|$ are polynomial related.
	
	Example: $\mathprob{SAT} \leq_p \mathprob{3SAT}$ is natural. 
	
	Question(s): What is output parameter $s_n$? Example?
\end{enumerate}

\newpage

\begin{lemma}
	\label{lem:qp-collapse}
	$\class{QP}^{\class{QP}} \subseteq \class{QP}$.
\end{lemma}

\begin{proof}
  Let $M$ be a TM dedicing a language $L \in \class{QP}^{\class{QP}}$ in
  quasi-polynomial time, say $\exp(\log^c n)$.
  We show that carrying out the oracle computation instead of calling the
  oracle does still guarantee a quasi-polynomial running time. To this end,
  let $M'$ be the TM deciding the $\class{QP}$-oracle, say in time
  $\exp(\log^{c'} m)$.
  Now, any input to $M'$ is at most of length $m = \exp(\log^c n)$. We
  therefore need time at most
  $\exp(\log^c (\exp(\log^c n))) = \exp(\log^{cc'} n)$
  for one oracle computation. At the same time, there are at most
  $\exp(\log^c n)$ calls, resulting in a total running time bound of
  $\exp(\log^c (\exp(\log^{cc'} n))) = \exp(\log^{c^2 c'} n)$
  which is still quasi-polynomial.
\end{proof}

% \begin{proof}
%   Let $M$ be a TM deciding a language $L \in \class{QP}^{\class{QP}}$.
%   
%   Then, $M$ runs in $O(n^{\log^c n})$ time with oracle access to a $\class{QP}$ machine.
%   
%   Suppose $f(n) = n^{\log^c n}$ and $g(n) = n^{\log^k n}$ are the run-time upper-bound functions of the ``main body'' of $M$ and the oracle machine respectively. This implies that the total running time of $M$ with some oracle access during the execution is bounded by $f(g(n))$ since $\class{QP}$ is a deterministic time related class.
%   
%   Let $t = \log^k n$, then $f(g(n)) = (n^{\log^k n})^{\log^c (n^{\log^k n})} = (n^t)^{(\log(n^t))^c} = (n^t)^{t^c \log^c n} = n^{t^{c+1}\log^c n}$
%   
%   Substitute $\log^k n$ to $t$, we have $f(g(n)) = n^{\log^{k(c+1)} n \log^c n} = n^{\log^{k(c+1) + c} n}$ which is quasi-polynomial. In other words, $M$ runs in quasi-polynomial time which implies that $L \in \class{QP}$.
%   
%   Hence, $\class{QP}^{\class{QP}} \subseteq \class{QP}$
%   
% \end{proof}

\begin{lemma}
  \label{lem:ph-sub-qp}
  If $\NP \subseteq \class{QP}$ then
  $\class{PH} \subseteq \class{QP}$.
\end{lemma}

\begin{proof}
	From the assumption $\NP \subseteq \class{QP}$, we have $\class{PH} = \NP^{\NP^{\NP^{\dots^{\NP}}}} \subseteq \class{QP}^{\class{QP}^{\class{QP}^{\dots^{\class{QP}}}}}$ (how do I put $k$ times in there?)
	
	Also, from Lemma 1, we have $\class{QP}^{\class{QP}^{\class{QP}^{\dots^{\class{QP}}}}} \subseteq \class{QP}^{\class{QP}^{\class{QP}^{\dots^{\class{QP}}}}} \subseteq \dots \subseteq \class{QP}$ (here, it means that we first have a stack of $k$ $\class{QP}$'s then it collapses to the stack of $k - 1$ and so on... thus, we finally collapses it to $\class{QP}$)
	
	Thus, $\NP \subseteq \class{QP} \implies \class{PH} \subseteq \class{QP}$
\end{proof}

\bigskip
\begin{proof} (alternative proof)
  We first show that $\NP = \exists \P \subseteq \class{QP}$ also implies that
  $\exists \class{QP} \subseteq \class{QP}$ by a padding
  argument.\footnote{The class $\exists \class{QP}$ is defined to consist of
  all languages $L$ such that there exists a quasi-polynomial TM $M$ and a
  polynomial $p$ with $x \in L \iff \exists y \in \{0,1\}^{p(|x|)}\
  M(x,y) = 1$}
  To this end, assume that $L \in \exists \class{QP}$ such that it is decided
  by a $\exp(\log^c n)$-time verifier $M$.
  We define our padded language as
  $L_{\textrm{pad}} \coloneqq \set{ x 1^{\exp(\log^c n)} \mid x \in L }$.
  We obtain that $L_{\textrm{pad}} \in \exists \P$ since applying $M$ to a
  padded element takes takes only linear time.
  It follows by our assumption that $L_{\textrm{pad}} \in \class{QP}$;
  say $L_{\textrm{pad}}$ is decided by an $\exp(\log^{c'} n)$-time TM $M'$.
  Now, the key idea lies in showing that this already implies
  $L \in \class{QP}$:
  Let $x \in \{0,1\}^*$ be of length $n \coloneqq |x|$ and
  its padded version $y \coloneqq x 1^{\exp(\log^c n)}$ of length
  $m = \exp(\log^c n)$. Then, the time needed by $M'$ to decide $y$ is
  \[
    \exp(\log^{c'} m)
    =
    \exp(\log^{c'}(\exp(\log^c n)))
    =
    \exp(\log^{c \cdot c'} n)
  \]
  which is again quasi-polynomial in $n$.
  Therefore, we can decide $x \in L$ by applying padding and running $M'$ all
  in quasi-polynomial time.

  The second step is to show that also
  $\forall \class{QP} \subseteq \class{QP}$.
  By definition, given a language $L \in \forall \class{QP}$, there exists a
  polynomial $p$ and a quasi-polynomial time verifier $M$ such that
  \[
    x \in L \iff \forall y \in \{0,1\}^{p(|x|)}\ M(x,y) = 1 \,.
  \]
  However, the right hand side is equivalent to the negation of
  $(\exists y \in \{0,1\}^{p(|x|)}\ M(x,y) = 0)$, a statement which can itself
  be decided in quasi-polynomial time by the above argument.
  Since all quasi-polynomial time TMs are also deterministic, we can also
  negate the output efficiently. It follows that $L \in \class{QP}$.

  By an inductive argument, we see that $\class{PH}$ collapses into a subset of
  $\class{QP}$.
\end{proof}

\begin{lemma}
  \label{lem:qp-super-circ}
  $\class{QP}^{\class{\Sigma}^p_k}$ contains a language which does not belong
  to $\Ppoly$ for some $k \in \N$.
\end{lemma}

\begin{proof}
  The proof follows a nonuniform diagonalization argument. We first define
  a language which will be hard to compute for any polynomial-size circuit
  family:
  Let $L'$ be the language consisting of tuples
  $\langle x, 1^{\exp(\log^3 n)} \rangle$
  with $n \coloneqq |x|$
  such that $C(x) = 1$ where $C$ is the
  lexicographically first circuit of size $\exp(\log^3 n)$ which is not
  computed by any circuit of size $\exp(\log^2 n)$.
  The existence of such a circuit for sufficiently large $n$ follows from a
  slightly more careful analysis of the nonuniform hierarchy theorem
  (Theorem~6.22).

  We can decide membership $\langle x, 1^{\exp(\log^3 n)} \rangle \in L'$
  by a $\class{\Sigma}^p_4$-oracle as in Problem~(1c) of Homework~7.
  Finally, we define our language $L$ of superpolynomial circuit complexity
  as the output of a $\class{QP}^{\class{\Sigma}^p_4}$-machine: Given an input
  $x \in \{0,1\}^n$, query the oracle for $L'$ with
  $\langle x, 1^{\exp(\log^3 n)} \rangle$ and output its answer.

  We constructed this language such that it is hard for a polynomial-size
  circuit to compute. To see this, assume to the contrary that there is
  a $n^a$-size circuit family.
  However, since $n^a \in o(\exp(\log^2 n))$ and we particularly excluded any
  circuits of size less than $\exp(\log^2 n)$, this is a contradiction.
\end{proof}

\begin{lemma}
  \label{lem:num-of-circs}
  There are at most $n^{\mathrm{polylog}(n)}$ different circuits of size
  $\log^c n$ for a constant $c$.
\end{lemma}

\begin{proof}
  In a circuit with $s \in \N$ many gates and inputs, each gate is connected to
  at most two out of $s$ gates, and computes one of the functions
  $\land, \lor, \neg$.
  This means, there are at most $3 \cdot s^2$ choices to construct each gate
  and thus $(3 s^2)^s$ choices to construct the whole circuit.
  Setting $s \coloneqq \log^c n$ gives us
  \[
    % O-notation missing....
    (3 \log^{2c} n)^{\log^c n}
    =
    \exp((\log^{2c} n) \cdot (\log^c n))
    =
    \exp(\log^{3c} n)
    =
    n^{\mathrm{polylog}(n)}
  \]
  many ways to construct a circuit of size $\log^c n$.
\end{proof}

\bigskip

\begin{theorem}[Theorem 15]
If $\mathprob{MCSP}$ is $\NP$-hard under a natural reduction from $\mathprob{SAT}$, then
\begin{enumerate} [1.]
	\item $\class{E}$ contains a family of Boolean functions $f_n$ not in $\Ppoly$ (i.o.), and
	\item $\class{E}$ contains a family of Boolean functions $f_n$ of circuit complexity $2^{\Omega(n)}$ (i.o.), unless $\NP \subseteq \class{SUBEXP}$
\end{enumerate}
\end{theorem}

\begin{proof}
\begin{enumerate}[-]
	\item Statement 1: consider two cases
	\begin{enumerate} [+]
		\item Case 1: $\NP \subseteq \class{QP}$
		

      Applying Lemma~\ref{lem:qp-collapse} and Lemma~\ref{lem:ph-sub-qp} to this assumption yields that
      $\class{QP}^{\class{PH}} \subseteq \class{QP}^{\class{QP}} \subseteq
       \class{QP} \subseteq \class{E}$.
      Lemma~\ref{lem:qp-super-circ} shows that $\class E$ also contains
      a language of superpolynomial circuit complexity.

      % how does
      %   'a language of superpolynomial circuit complexity'
      % relate to
      %   'a family of Boolean functions not in P/poly'
      % ?
		
		Hence, $\class{E} \not\subseteq \Ppoly$\\
		
		\item Case 2: $\NP \not\subseteq \class{QP}$
		
      We pick any set $U$ of unsatisfiable Boolean formulae such that for each
      formula $\phi_k \in U$ in $k$ variables, its size $n = |\phi|$ is related
      to the number of variables by $k \in \Theta(\log n)$. Say,
      $U \coloneqq \set{ \phi_k \mid k \in \N }$
      where
      \[
        \phi_k(x_1, \dots, x_k) \coloneqq
        (x_1 \land \bar x_1) \land
        \underbrace{(\cdots \cdots\cdots \cdots \cdots \cdots\cdots \cdots)}_{
          \textrm{arbitrary term of size $2^k$}} \,.
      \]
      Now based on this, we apply $R$ to define our language: Let
      $\langle T_k, s \rangle \coloneqq R(\phi_k)$ for each $\phi_k \in U$
      and define
      \[
        L \coloneqq
        \set{
          x \in \{0,1\}^k \mid
          k \in \N,
          T_k(x) = 1
        } \,.
      \]
      We obtain the following.
      \begin{enumerate}[(i)]
        \item $L \in E$:
          %
          We can easily construct a machine running to decide $L$ in time
          $2^{O(n)}$: Given an input $x \in \{0,1\}^k$, first construct the
          Boolean formula $\phi_k \in U$ corresponding to this input length.
          Note that $k \in \Theta(\log n)$ implies
          that $n \in 2^{O(k)}$. % why would that be?
          In particular, this means that we can write down $\phi_k$ in time
          $2^{O(k)}$.
          %
          Next, we apply our natural reduction $R$ to $\phi_k$ to obtain
          $\langle T_k, s \rangle = R(\phi_k)$.
          As $R$ is polynomial-time, say $n^a$, the time needed for this
          application is also bound by $(2^{O(k)})^a = 2^{O(k)}$.
          %
          In a final step, we output $T_k(x)$.
          %
          Overall, we can carry out the decision in time $2^{O(k)}$ proving
          that $L \in E$.

        \item $L \not\in \Ppoly$:
          %
          We assume to the contrary that $L \in \Ppoly$.
          Remember that the reduction $R$ is natural. In particular, this means
          for a mapping $R(\phi_k) = \langle T_k, s \rangle$ that
          the parameter $s$ depends only on a polynomial in $k$, say it is
          upper bounded by
          $s = k^b = \log^b n$ % more like \Theta(...)
          for a $b \in \N$.
          However, Lemma~\ref{lem:num-of-circs} implies that there are at
          most $n^{\mathrm{polylog}(n)}$ circuits of size $s$.
          We can simply enumerate these and check for each circuit whether
          it represents $T_k$ to decide
          $\langle T_k , s\rangle \in \mathprob{MCSP}$.
          Note that testing whether a circuit $C$ of size $s$ represents
          $T_k$ only requires us to do $2^k$ evaluations of $C$, each of which
          take time $O(s)$.
          Overall, we can decide membership in quasi-polynomial time.
          This, in turn, allows us to decide $\phi \in \SAT$ by applying $R$
          and carrying out the above enumeration,
          implying that $\SAT \in \class{QP}$.
          We have, however, assumed that $\NP \not\subseteq \class{QP}$, which
          is a contradiction.
      \end{enumerate}


	\end{enumerate}
	\item Statement 2: the proof is similar (you want to do it too?)
    % I believe Statement is ruled out of Case 1 since if assuming
    % NP <= QP implies
    % (since QP <= SUBEXP) that NP <= SUBEXP.
    % So that should rather be a bullet point in Case 2
\end{enumerate}
\end{proof}

\subsection{Natural and Unnatural Reductions}
Consider the following reduction from
$\mathprob{Clique}$\footnote{
  The clique problem ($\mathprob{Clique}$) asks for an instance
  $\langle G, n \rangle$ of a graph $G$ and integer $n$, whether $G$ has
  a completely connected subgraph of size $n$.
}
to
$\mathprob{SubIso}$:\footnote{
  The subgraph isomorphism problem ($\mathprob{SubIso}$) asks, given two graphs
  $G$ and $H$, whether there is a subgraph $G'$ of $G$ which is isomorphic to
  $H$.
}
\[
  f(G, n) =
  \left\{
    \begin{array}{ll}
      \langle K_1, K_2 \rangle & \textrm{if } |V(G)| < n \\
      \langle G, K_{n} \rangle & \textrm{otherwise} \\
    \end{array}
  \right.
\]
where $K_n$ is the complete graph on $n$ vertices.
Note that the reduction is correct since $G$ has a clique of size $n$ if and
only if it has a connected subgraph $K_n$. Also, $G$ cannot have a clique of
size $n$ if it does not even have $n$ vertices.
Also, the reduction can be carried out in polynomial time: In fact,
we can commpute the output size as
\[
  |f(G,n)| = 
  \left\{
    \begin{array}{ll}
      O(1) & \textrm{if } |V(G)| < n \\
      \Theta(|G|) & \textrm{otherwise} \\
    \end{array}
  \right.
\]
However, $|f(G,n)|$ is certainly not a function in the input size $(|G|+n)$,
so this reduction is not natural.

With a little more care, we can turn this reduction into a natural reduction.
For example, we can define
\[
  f'(G, n) =
  \left\{
    \begin{array}{ll}
      \langle G \cup K_1, G \rangle & \textrm{if } |V(G)| < n \\
      \langle G, K_{n} \rangle & \textrm{otherwise} \\
    \end{array}
  \right.
\]
where $G \cup H$ denotes the union of two graphs $G$ and $H$.
Now, the output size becomes
$|f'(G,n)| = \Theta(|G|)$
in both cases.


\end{document}






