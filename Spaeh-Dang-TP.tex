%
% To use this as a template for turning in your solutions, change the flag
% \inclsolns from 0 to 1. Make sure you include macros.tex in the directory
% containing this file. Edit the "author" and "collaborators" fields as
% appropriate. Write your solutions where indicated.
%

\documentclass[11pt]{article}

\usepackage{fullpage}
\usepackage{graphicx}
\usepackage{enumerate}
\usepackage{comment}
\usepackage{amsmath,amssymb,amsthm}
\usepackage{hyperref}
\usepackage[capitalise]{cleveref}

\include{macros}

%If you want to add more macros, do it like this
\newcommand{\dolphin}{\mathsf{DOLPHIN!!}}


\author{Your name, should you choose to include it}
\title{Silly title of your review: \\ A serious title on an extra line if you need it}
\date{November 17, 2020}

\begin{document}
	\maketitle
	
\section{Introduction}

What are the main contributions of the paper you are reviewing? This introduction should emphasize conceptual contributions and why they are important, avoiding technical definitions and details.

\section{Preliminaries}

Introduce important notation and definitions, like this:

\begin{definition}
	The class $\class{PP}$ consists of all languages $L$ such that there exists a probabilistic Turing machine $M$ where $x \in L$ iff $\Pr[M(x) = 1] \ge 1/2$.
\end{definition}

\section{Main Results}

Next, you should describe the main results of the paper at a technical level, giving precise theorem statements. Then you should choose one of these main results and give a detailed proof. Depending on the topic you choose, it may be difficult to do this concisely. If that is the case, you may decide to prove a special case of the result (that illustrates most of the main ideas). You might also ``black-box out'' technical lemmas that require lengthy calculations but are not so conceptually important, so long as these lemmas are stated precisely and you can prove how they fit together.

This is a theorem due to Beigel, Reingold, and Spielman~\cite{brsconf}. (The reference for the journal version of the paper is~\cite{brsjournal}.)

\begin{theorem}
	$\class{PP}$ is closed under intersection. That is, if $L_1, L_2 \in \PP$, then $L_1 \cap L_2 \in \PP$.
\end{theorem}

\begin{proof}
	I'm not gonna do this.
\end{proof}

\section{Conclusion}

Further discussion of the implications of the results you've described and directions for future research. 

% I like to use bibtex to organize reference, but you can use whatever you like. https://dblp.org/ is a good source for bibtex entries for CS papers
\bibliographystyle{alpha}
\bibliography{term_paper_bib}

\end{document}



